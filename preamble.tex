% ##################################################
% ENCODING
% ##################################################
\usepackage{cmap}               % PDF character encoding
\usepackage[T1]{fontenc}        % 8-bit font encoding
\usepackage[utf8]{inputenc}     % UTF-8 input encoding

% ##################################################
% LANGUAGE
% ##################################################
% \usepackage[ngerman]{babel} % Uncomment if you write in German
\usepackage{csquotes}
\usepackage{blindtext}

% ##################################################
% DOCUMENT VARIABLES
% ##################################################

% -------------
% Document data
% -------------
\newcommand{\docDeadline}{Submission Date}
\newcommand{\docKeywords}{These, Are, Your, Keywords}

% ---------------
% University Data
% ---------------
\newcommand{\uniTitle}{Your University}
\newcommand{\uniPostalCode}{12345}
\newcommand{\uniCity}{Fantasy City}
\newcommand{\uniStreet}{Fantasy Street 123}

% -------
% Authors
% -------
\newcommand{\docAuthors}{John Doe, Jane Doe, Catherine Parr}

% ##################################################
% PAGE FORMATTING
% ##################################################

% -----------
% Page layout
% -----------
\usepackage[
    paper=a4paper,
    inner=1cm,
    outer=1cm,
    top=1cm,
    bottom=1.5cm,
]{geometry}

% See https://ctan.mc1.root.project-creative.net/macros/latex/required/tools/multicol.pdf
\usepackage{multicol}

% ------------------
% Section formatting
% ------------------

% See https://ctan.kako-dev.de/macros/latex/contrib/titlesec/titlesec.pdf
\usepackage{titlesec}

% Adjust Section style and spacing
\titleformat{\section}{\bfseries\sffamily\raggedright}{\thesection .}{0.5em}{}
\titlespacing\section{0pt}{12pt plus 4pt minus 2pt}{1pt plus 1pt minus 1pt}

% Adjust Subsection style and spacing
\titleformat{\subsection}{\bfseries\sffamily\raggedright}{\thesubsection}{0.5em}{}
\titlespacing\subsection{0pt}{12pt plus 4pt minus 2pt}{0pt plus 2pt minus 1pt}

% Adjust Subsubsection style and spacing
\titleformat{\subsubsection}[runin]{\bfseries\sffamily}{\thesubsubsection}{0.5em}{}[.\hspace*{0.5em}]
\titlespacing\subsubsection{\parindent}{6pt plus 3pt minus 2pt}{0pt plus 2pt minus 1pt}

% ---------------
% Footer / Header
% ---------------

% See https://ftp.agdsn.de/pub/mirrors/latex/dante/macros/latex/contrib/fancyhdr/fancyhdr.pdf
\usepackage{fancyhdr}              % Use fancyhdr package
\pagestyle{fancy}                  % Activate custom styling
\fancyhf{}                         % Clear everything
\fancyfoot[C]{\thepage}            % Display page number centered in the footer
\renewcommand{\footrulewidth}{1pt} % Display thin footer line
\renewcommand{\headrulewidth}{0pt} % Display no header line

% ##################################################
% BIBLIOGRAPHY 
% ##################################################

% ---------
% Citations
% ---------
\usepackage{biblatex}                       % Better citation
\addbibresource{bibliography/online.bib}    % Include online citations (websites), for articles @article can be used
\addbibresource{bibliography/standards.bib} % Include various web / internet standards

% --------
% Acronyms
% --------

% For glossary styles see: https://www.dickimaw-books.com/gallery/glossaries-styles/
\usepackage[
    acronym,     % Create glossary of acronyms
    nopostdot,   % Remove dot after acronym definition
    nonumberlist % Disable numbered lists
]{glossaries}
\renewcommand*{\glsnamefont}[1]{\textbf{\sffamily #1}} % Change font to sans-serif
\makeglossaries                                        % Generate glossaries 
\newacronym{api}{API}{Application Programming Interface}

\newacronym{http}{HTTP}{Hyper Text Transfer Protocol}
\newacronym{html}{HTML}{Hyper Text Markup Language}                          % Define input file

% ##################################################
% IMAGES AND FIGURES
% ##################################################
\renewcommand{\thefigure}{\arabic{figure}} % simple numbering without chapter
\usepackage{graphicx}                      % support for including images
\graphicspath{{img/}}                      % default path

% ##################################################
% TABLES
% ##################################################
\usepackage{tabularx}
\usepackage{makecell}

% ##################################################
% CUSTOM ENVIROMENTS
% ##################################################

% Paper envioment wraps text in a two-column multicol
\newenvironment{paper}{%
    \begin{multicols}{2}%
        }{%
    \end{multicols}%
}

% Overwrites existing abstract environment
\DeclareDocumentEnvironment{abstract}{}{%
    \section*{\abstractname}\itshape
}
{\upshape \sffamily \textbf{Keywords: \docKeywords}
}

% ##################################################
% CUSTOM COMMANDS
% ##################################################
\usepackage{etoolbox} % Multiple tools like loops
\usepackage{intcalc}  % Expandable arithmetic operations
\usepackage{xparse}   % Parse lists

% -------
% Helpers
% -------

% If the provided string is not blank (empyt or only spaces) parameter #2 is displayed, otherwise #3
\newcommand{\ifnotblank}[3]{
    \expandafter\notblank\expandafter{#1}{#2}{#3}%
}

\newcommand{\docTitle}{This is the default title. Plese change me via \textbackslash setTitle\{~\}}
\newcommand{\setTitle}[1]{
    \ifnotblank{#1}{%
        \renewcommand{\docTitle}{#1}
        \hypersetup{
            pdftitle ={\docTitle}
        }
    }{}
}

\newcommand{\docSubTitle}{}
\newcommand{\setSubtitle}[1]{
    \ifnotblank{#1}{%
        \renewcommand{\docSubTitle}{#1}
    }{}
}

\newcommand{\docType}{}
\newcommand{\setDoctype}[1]{
    \ifnotblank{#1}{%
        \renewcommand{\docType}{#1}
    }{}
}

% -------
% Authors
% -------

% Variables
\newcommand{\authorList}{}
\newcommand{\currentAffiliation}{}
\newcounter{totalAuthorCount}
\newcounter{authorCounter}
\setcounter{authorCounter}{1}
\newcounter{authorIndex}

% \setAuthors let's the user set all authors as a comma-separated list
\newcommand{\setAuthors}[1]{
    \renewcommand{\authorList}{#1}
    \expandafter\forcsvlist\expandafter\countAuthors\expandafter{\authorList}
}

% \countAuthors counts the total amount of authors
\newcommand{\countAuthors}[1]{
    \stepcounter{totalAuthorCount}
}

% \printAuthors let's the user print out the author list as a grid table
\newcommand{\printAuthors}{
    \begin{center}
        \begin{tabularx}{0.9\textwidth}{ X X X }
            \expandafter\internalPrintAuthors\expandafter{\authorList}
        \end{tabularx}
    \end{center}
}

% \internalPrintAuthors iterates the list via \ProcessList
\NewDocumentCommand{\internalPrintAuthors}{ >{\SplitList{,}} m }{%
    \ProcessList{#1}{\printAuthorTriple}
}

% \printAuthorTriple prints the <Author, E-Mail, Affiliation> triple
\NewDocumentCommand{\printAuthorTriple}{ >{\SplitList{,}} m }{%
    \ifthenelse{\equal{\intcalcMod{\value{authorCounter}}{3}}{0}}{%
        \makecell{
            \ProcessList{#1}{\splitAuthorTriple}
        }
        \ifthenelse{\value{authorCounter}<\value{totalAuthorCount}}{\\}{}
    }{%
        \makecell{
            \ProcessList{#1}{\splitAuthorTriple}
        }
        \ifthenelse{\(\value{totalAuthorCount}>1 \AND \value{authorCounter}<\value{totalAuthorCount}\)}{&}{}
    }%
    \stepcounter{authorCounter}
}

% \splitAuthorTriple splits up the author, e-mail address and affiliation
\newcommand{\splitAuthorTriple}[1]{
    \ifthenelse{\equal{\value{authorIndex}}{0}}{%
        \printAuthorName{#1}%
        \stepcounter{authorIndex}%
    }{%
        \ifthenelse{\equal{\value{authorIndex}}{1}}{%
            \ifthenelse{\equal{#1}{-}}{\\}{%
                \printAuthorAffiliation{#1}%
            }
            \stepcounter{authorIndex}%
        }{%
            \printAuthorEmail{#1}
            \setcounter{authorIndex}{0}%
        }%

    }%
}

\newcommand{\printAuthorName}[1]{\sffamily \bfseries #1}
\newcommand{\printAuthorEmail}[1]{\small \mdseries #1}
\newcommand{\printAuthorAffiliation}[1]{\textsuperscript{#1} \\}

% -----------
% Affiliation
% -----------
\newcommand{\affiliationList}{}
\newcounter{totalAffiliations}
\newcounter{affiliationCounter}

\newcommand{\setAffiliations}[1]{%
    \renewcommand{\affiliationList}{#1}
    \expandafter\forcsvlist\expandafter\countAffiliations\expandafter{\affiliationList}
}

\newcommand{\countAffiliations}[1]{
    \stepcounter{totalAffiliations}
}

\newcommand{\printAffiliations}{%
    \ifnotblank{\affiliationList}{%
        \begin{center}
            \begin{tabularx}{0.9\textwidth}{ X }
                \expandafter\internalPrintAffiliations\expandafter{\affiliationList}
            \end{tabularx}
        \end{center}
    }{}
}

\NewDocumentCommand{\internalPrintAffiliations}{ >{\SplitList{,}} m }{%
    \ProcessList{#1}{\printAffiliation}
}

\newcommand{\printAffiliation}[1]{
    \stepcounter{affiliationCounter}
    \makecell{
        \sffamily \small \textsuperscript{\number\value{affiliationCounter}}#1
    }
    \ifthenelse{\value{affiliationCounter}<\value{totalAffiliations}}{\\}{}
}

% ##################################################
% PDF SETTINGS
% ##################################################
% Do this as the last step beause the variables are set above
\usepackage[
    colorlinks=true,
    linkcolor=black,
    citecolor=black,
    filecolor=black,
    urlcolor=black,
    bookmarks=true,
    bookmarksopen=true,
    bookmarksopenlevel=3,
    bookmarksnumbered,
    plainpages=false,
    pdfpagelabels=true,
    hyperfootnotes
    % pdftitle ={\docTitle},
    % pdfauthor={\docAuthors},
    % pdfcreator={\docAuthors}
]{hyperref}